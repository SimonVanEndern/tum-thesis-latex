% !TeX root = ../main.tex
% Add the above to each chapter to make compiling the PDF easier in some editors.

\chapter{Introduction}\label{chapter:introduction}

The introduction is meant to motivate the subject area (why is this important?),
define the problem you are interested in (what are you doing?), and limit the
scope (where do you stop?). It also gives an outline of the thesis (which chapters
will explain what?) and explains how you are going the approach your subject.
\\

With the advent of the internet and large-scale applications, the question of privacy has drawn increasing attention.
Especially with services like Twitter, Facebook, Google \& Co. there are problems and privacy infringements when user data is realeased.
One example is that the location data of Twitter tweets was published without asking the user for permission. Furthermore this data is only available through the API,
so that the user is not aware of this infringement. Using this data, ~\parencite{twitter} has shown that this data can be used to infer a users home address and often also the work address, even if the user itself is privacy-aware, thus does not publish his / her name, etc.


~\parencite{cellphone} finds that even when personal data is anonymized thus that names and addresses, etc. are removed, sensitive information can be inferred from the data.
In this study it was shown that from call-records in the US the home address and also often the work address of a person could be inferred.
They highlight that while adhering to the k-anomymity model proposed by ~\parencite{k-anonymity} it is practically not possible to publish datasets that are still of any significant use.
\\

We will use the definition of location privacy as defined by ~\parencite{location-privacy}: " the ability to prevent other parties from learning
one’s current or past location". They further propose a different approach to preserve privacy. TODO!!!

Also ~\parencite{privacy-home-work-pairs} highlights the thread that home and work locations can be inferred from anonymized datasets and can in combination with other sources yield even more information about a user. To reduce this risk, they propose "to collect the minimum amount of information needed". In contrary, we want to investigate another approach, so that rich data can still be used and be published in an aggregated manner to let people profit from the data but still preserve privacy.

This research shows that publishing raw data is critical, even when the data is anonymized.
As still this data could be useful for many stakeholders, we will investigate how on the one hand aggregated datacan be published without imposing any privacy risk to the owners of the data and on the other hand develop a prototype of a mobile application through which this location data is aggregated in a decentralized manner so that the raw user data never leaves the users' device.

Another problem that arises is that anonymization algorithms applied to datasets prior to publishing them might yield good results if the location data is in a densily populated area but might perform poorly if the population is only sparse \parencite{privacy-in-sparse-populated-locations}.

\section{Existing approaches}

\begin{itemize}
  \item Collect less data \parencite{privacy-home-work-pairs}
  \item Mixing approach \parencite{location-privacy}
  \item Anonymize data to meet the kriteria of k-anonymity \parencite{k-anonymity} and \parencite{cellphone}
  \item spatial cloaking \parencite{krumm}
\end{itemize}


\section{Section}
Citation test~\parencite{latex}.

\subsection{Subsection}

See~\autoref{tab:sample}, \autoref{fig:sample-drawing}, \autoref{fig:sample-plot}, \autoref{fig:sample-listing}.

\begin{table}[htpb]
  \caption[Example table]{An example for a simple table.}\label{tab:sample}
  \centering
  \begin{tabular}{l l l l}
    \toprule
      A & B & C & D \\
    \midrule
      1 & 2 & 1 & 2 \\
      2 & 3 & 2 & 3 \\
    \bottomrule
  \end{tabular}
\end{table}

\begin{figure}[htpb]
  \centering
  % This should probably go into a file in figures/
  \begin{tikzpicture}[node distance=3cm]
    \node (R0) {$R_1$};
    \node (R1) [right of=R0] {$R_2$};
    \node (R2) [below of=R1] {$R_4$};
    \node (R3) [below of=R0] {$R_3$};
    \node (R4) [right of=R1] {$R_5$};

    \path[every node]
      (R0) edge (R1)
      (R0) edge (R3)
      (R3) edge (R2)
      (R2) edge (R1)
      (R1) edge (R4);
  \end{tikzpicture}
  \caption[Example drawing]{An example for a simple drawing.}\label{fig:sample-drawing}
\end{figure}

\begin{figure}[htpb]
  \centering

  \pgfplotstableset{col sep=&, row sep=\\}
  % This should probably go into a file in data/
  \pgfplotstableread{
    a & b    \\
    1 & 1000 \\
    2 & 1500 \\
    3 & 1600 \\
  }\exampleA
  \pgfplotstableread{
    a & b    \\
    1 & 1200 \\
    2 & 800 \\
    3 & 1400 \\
  }\exampleB
  % This should probably go into a file in figures/
  \begin{tikzpicture}
    \begin{axis}[
        ymin=0,
        legend style={legend pos=south east},
        grid,
        thick,
        ylabel=Y,
        xlabel=X
      ]
      \addplot table[x=a, y=b]{\exampleA};
      \addlegendentry{Example A};
      \addplot table[x=a, y=b]{\exampleB};
      \addlegendentry{Example B};
    \end{axis}
  \end{tikzpicture}
  \caption[Example plot]{An example for a simple plot.}\label{fig:sample-plot}
\end{figure}

\begin{figure}[htpb]
  \centering
  \begin{tabular}{c}
  \begin{lstlisting}[language=SQL]
    SELECT * FROM tbl WHERE tbl.str = "str"
  \end{lstlisting}
  \end{tabular}
  \caption[Example listing]{An example for a source code listing.}\label{fig:sample-listing}
\end{figure}
