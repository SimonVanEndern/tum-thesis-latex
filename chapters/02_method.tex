\chapter{Related Work}\label{chapter:related-work}

\parencite{location-privacy} introduces the concept of mix-nodes already known from privacy research on a network level (TODO: "copy" related work part of paper "time-to-confusion"). They propose a framework in which privacy is protected through frequently changing pseudonyms. Furthermore they find that similarly to the problem of identifying consecutive samples in \parencite{time-to-confusion}, the change of pseudonyms has also to be obfuscated in order to provide complete privacy. In contrast, this paper focuses mostly on solving the problem that location aware services that e.g. notify you when you are close to a venue of interest, do not need to have access to your location data at anytime but can register to events with a mix-node. Thus they register for the venues of interested and only get notified when the mix-node, which is trusted and has complete access to location data, detects a match. One sees straight away, that this again depends on trust of the users on the mix-node. Nevertheless, the proposed solution of mix-nodes and mix-zones analyzed on a sample shows that even using this framework, privacy cannot be provided, especially as here again the entropy provided by the history of the released or somehow collected data-set makes it too hard to obfuscate the consecutiveness of different pseudonyms.

\parencite{k-anonymity} is the current state of the art of minimum data protection. They define a dataset as the commonly understood tables in SQL. Besides the unique identifier used in the table, a quasi-identifier is the combination of several attributes with which a set of entries can be identified. a dataset adheres to the rules of k-anonymity, if querying every possible such identifier returns at lest a set of k different entries. Thus 1-anonymity identifies an entry exactly and provides no anonymity at all. The anonymity problem arises not from the dataset itself, but from a combination of datasets, that have the attributes of the quasi-identifier in common. This way anonymous knowledge from both datasets can be linked in order to infer information not intended to be made public. They also highlight, that also publishing the same dataset with different privacy-rules, i.e. different anonymization techniques applied, can result in inferences that reveal the original dataset.

\parencite{k-anonymity} clearly highlights that there are two approaches to hiding sensitive information. One is to restrict queries to a database that might reveal sensitive information. In contrast to this approach, they focus on anonymizing the data already before any access to it. Nevertheless, this is based on the assummption that the data owner knows about any possible quasi-identifier in order to obfuscate the dataset sufficiently to provide k-anonymity for all quasi-identifiers. If one quasi-identifier is not thought of, the dataset might expose 1-anonymity for this identifier and result in possible exposures of data not intended to be public.

\parencite{k-anonymity} also discusses further problems that are easy to tackle but nevertheless necessary to protect users' privacy. The order of the published table must be random. Otherwise there is more information (hidden) available that can be used to break k-anonymity. Another problem is when the same table is released and obfuscated differently for the same quasi-identifier, other attributes in the releases can be used to link entries and thus de-anonymize the data.

\parencite{privacy-home-work-pairs} further investigates the fact that from a dataset containing GPS data of trajectories or e.g. twitter-posts as in \parencite{twitter} the home location can be inferred with high probability. They show that also the work location can be identified with pretty high accuracy and probability. Furthermore they find that people who live and work in different regions or more generally, the further work and home diverge, the smaller the anonymity set of the specific user in the dataset and thus the lower also the anonymity. This is similar to the findings of \parencite{location-privacy} that users in less populated areas are exposed to more privacy risk than in denser areas.

\parencite{mix-zones} extends the analyzis of \parencite{location-privacy}.

TODO: Cite middleware usage approach by \parencite{gruteser2003anonymous}
TODO: Cite approach of disclosure algorithms by \parencite{gruteser2005anonymity}
TODO: Cite confusion approach similar to \parencite{time-to-confusion} by \parencite{hoh2005protecting}
TODO: Cite querying an anonymization by \parencite{mokbel2006new}
TODO: Read \parencite{tang2006putting}