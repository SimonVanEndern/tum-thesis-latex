% !TeX root = ../main.tex
% Add the above to each chapter to make compiling the PDF easier in some editors.
\chapter{Related Work}\label{chapter:related-work}
\section{Inferring data from already published datasets}

\parencite{krumm} is one of the first investigating privacy issue in location data. For inferring the home location of a set of car travelling traces of their research subjects, they identify taking the last destination of the car before 3 pm as the most successful among 4 algorithms / heuristics to determine a persons home location. They where able to identify 12.8\% of the users home coordinates. Furthermore by looking up those home coordinates on a free online tool, they are able to retrieve the correct name for about 5\% of the subjects. Nevertheless, those results could be improved by far, as they show that the used data source / white pages are outdated.
In order to protect anonymity, they mainly identify the following different countermeasures:
\begin{itemize}
	\item Pseudonymity: Stripping original IDs from the dataset (by many shown not to be sufficient)
	\item Spatial Cloaking: Application of k-anonymity by hiding all data points in a circle with the center placed randomly around the actual home address
	\item Noise: Adding Gaussian noise to each data point
	\item Rounding: Placing a grid on the location data and mapping each data point to the closest intersection
	\item Dropped Samples: Completely dropping samples in order to reduce the frequency of the data points.
\end{itemize}
Of the application of these countermeasures, only spatial cloaking can preserve data quality, while a Noise with a standard deviation of 5km and a grid for rounding with 5km distances is needed, which render the data useless for many applications.
On the contrary, they showed how easy it is, to make the final step from home location to actual identity. Furthermore, there analyzis is only based on data covering two weeks.


\parencite{cellphone} finds that even when personal data is anonymized thus that names and addresses, etc. are removed, sensitive information can be inferred from the data.
In this study it was shown that from call-records in the US the home address and also often the work address of a person could be inferred.
They highlight that while adhering to the k-anomymity model proposed by ~\parencite{k-anonymity} it is practically not possible to publish datasets that are still of any significant use.
\\

Also \parencite{privacy-home-work-pairs} highlights the thread that home and work locations can be inferred from anonymized datasets and can in combination with other sources yield even more information about a user. To reduce this risk, they propose "to collect the minimum amount of information needed". In contrary, we want to investigate another approach, so that rich data can still be used and be published in an aggregated manner to let people profit from the data but still preserve privacy.


Another problem that arises is that anonymization algorithms applied to datasets prior to publishing them might yield good results if the location data is in a densily populated area but might perform poorly if the population is only sparse \parencite{time-to-confusion}.

\parencite{time-to-confusion} identify that while privacy algorithms might successfully provide privacy for location data samples in highly frequented areas, but perform poorly and disclose sensitive information for samples in areas with lower traffic frequency. They discuss the problem commonly accepted in research that either the quality of the data becomes poor or useless when applying techniques like k-anonymity \parencite{k-anonymity-old, k-anonymity, k-anonymity-achieving} or that privacy cannot be guaranteed. They propose a novel algorithm based on time-to-confusion. Thus basically whenever it is possible to attribute two different samples of a dataset with a high probability to the same user, the corresponding sample gets removed from the data-set to be published. This is necessary, as "the degree of privacy risk strongly depends on how long an adversary can follow a vehicle" \parencite{time-to-confusion}. In more detail, time-to-confusion also takes into account the entropy information provided by the whole dataset, thus that even when two samples cannot be connected with high probability due to to many possible consecutive samples, analyzing the whole dataset can provide information that actually the possible consecutive samples have different probabilities due to common route choices. E.g. a vehicle on a highway is much more likely to follow on the highway for some more time than leaving the highway. While this information is taken into account, they point out the limitations of their work that when the dataset is matched with street maps, even more samples would have to be remoed to ensure privacy because it will render some former possible consecutive samples impossible due to missing streets connecting them. 

\parencite{location-privacy} introduces the concept of mix-nodes already known from privacy research on a network level (TODO: "copy" related work part of paper "time-to-confusion"). They propose a framework in which privacy is protected through frequently changing pseudonyms. Furthermore they find that similarly to the problem of identifying consecutive samples in \parencite{time-to-confusion}, the change of pseudonyms has also to be obfuscated in order to provide complete privacy. In contrast, this paper focuses mostly on solving the problem that location aware services that e.g. notify you when you are close to a venue of interest, do not need to have access to your location data at anytime but can register to events with a mix-node. Thus they register for the venues of interested and only get notified when the mix-node, which is trusted and has complete access to location data, detects a match. One sees straight away, that this again depends on trust of the users on the mix-node. Nevertheless, the proposed solution of mix-nodes and mix-zones analyzed on a sample shows that even using this framework, privacy cannot be provided, especially as here again the entropy provided by the history of the released or somehow collected data-set makes it too hard to obfuscate the consecutiveness of different pseudonyms.

\parencite{k-anonymity} is the current state of the art of minimum data protection. They define a dataset as the commonly understood tables in SQL. Besides the unique identifier used in the table, a quasi-identifier is the combination of several attributes with which a set of entries can be identified. a dataset adheres to the rules of k-anonymity, if querying every possible such identifier returns at lest a set of k different entries. Thus 1-anonymity identifies an entry exactly and provides no anonymity at all. The anonymity problem arises not from the dataset itself, but from a combination of datasets, that have the attributes of the quasi-identifier in common. This way anonymous knowledge from both datasets can be linked in order to infer information not intended to be made public. They also highlight, that also publishing the same dataset with different privacy-rules, i.e. different anonymization techniques applied, can result in inferences that reveal the original dataset.

\parencite{k-anonymity} clearly highlights that there are two approaches to hiding sensitive information. One is to restrict queries to a database that might reveal sensitive information. In contrast to this approach, they focus on anonymizing the data already before any access to it. Nevertheless, this is based on the assummption that the data owner knows about any possible quasi-identifier in order to obfuscate the dataset sufficiently to provide k-anonymity for all quasi-identifiers. If one quasi-identifier is not thought of, the dataset might expose 1-anonymity for this identifier and result in possible exposures of data not intended to be public.

\parencite{k-anonymity} also discusses further problems that are easy to tackle but nevertheless necessary to protect users' privacy. The order of the published table must be random. Otherwise there is more information (hidden) available that can be used to break k-anonymity. Another problem is when the same table is released and obfuscated differently for the same quasi-identifier, other attributes in the releases can be used to link entries and thus de-anonymize the data.

\parencite{privacy-home-work-pairs} further investigates the fact that from a dataset containing GPS data of trajectories or e.g. twitter-posts as in \parencite{twitter} the home location can be inferred with high probability. They show that also the work location can be identified with pretty high accuracy and probability. Furthermore they find that people who live and work in different regions or more generally, the further work and home diverge, the smaller the anonymity set of the specific user in the dataset and thus the lower also the anonymity. This is similar to the findings of \parencite{location-privacy} that users in less populated areas are exposed to more privacy risk than in denser areas.

\parencite{mix-zones} extends the analyzis of \parencite{location-privacy}.

TODO: Cite middleware usage approach by \parencite{gruteser2003anonymous}
TODO: Cite approach of disclosure algorithms by \parencite{gruteser2005anonymity}
TODO: Cite confusion approach similar to \parencite{time-to-confusion} by \parencite{hoh2005protecting}
TODO: Cite querying an anonymization by \parencite{mokbel2006new}
TODO: Read \parencite{tang2006putting}

\section{Approaches to avoid central datasets}

\parencite{p2p-android} addresses the possible solution of p2p communication instead of using a central instance. They state, that mobile p2p communication is mainly based on WIFI and bluetooth. They propose a middleware embedded on top of the android operating system to facilitate widespread use of p2p. However, those p2p networks are so far limited to devices close to each other locally, as it works over WIFI or Bluetooth and so far there is no established approach to connect smaller local p2p networks over the internet to completely stop relying on central server instances.