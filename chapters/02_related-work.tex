% !TeX root = ../main.tex
% Add the above to each chapter to make compiling the PDF easier in some editors.
\chapter{Related Work}\label{chapter:related-work}

\section{Classification of Location Data Usage according to Acceptable Delay}
In order to review existing approaches and research, we classify location-aware services by the acceptable delay of the location information being available. Similar to the classification implied by Hoh et al. \parencite{hoh2005protecting} and Gruteser et al. \parencite{gruteser2003anonymous}, we define three categories:
\begin{enumerate}
  \item \textbf{Almost no delay tolerance:} e.g. an application showing a pop-up about a nearby venue e.g. a coffee shop when a pedestrian passes by.
  \item \textbf{Some delay tolerance e.g. one minute:} An application e.g. google maps derives the information of congested traffic from devices reporting their GPS data which show lower than usual speed on the respective road. As congestions worth reporting last longer than one minute, some delay in the device's information reaching the server is acceptable.
  \item \textbf{Significant delay tolerance of hours, days or even weeks:} For historical and statistical use of location data e.g. to find out about popular visiting times of a venue, almost any delay is acceptable.
\end{enumerate}
Some research deals with category one where almost no delay tolerance is acceptable \parencite{location-privacy, mix-zones, tang2006putting}. For example, Tang et al. \parencite{tang2006putting} propose a possibility to collect data about the number of people currently at a venue through the number of devices registered in the respective WIFI. This data is collected by a client application and anonymously send to a central server. Mokbel et al. \parencite{casper} and Tang et al. \parencite{tang2006putting} propose a solution where not the exact location is sent to a server but the rough region of the user. The server then sends a list of all possible matches e.g. petrol stations in this area to the client. Locally, this list is then matched with the exact location in order to fulfil the aim of the respective application.
Most research though investigates users' privacy for category 3 where the delay of the data being available for processing is not an issue \parencite{krumm, cellphone, privacy-home-work-pairs, twitter}. In the following, we will concentrate on this case as well.
%review research tackling location privacy in case 3 and 2 and then briefly point out the findings for case 1.

\section{Privacy Problems Arising from Location Data}
\subsection{Risk of Privacy Intrusion through Theft of Central Databases}
Centralized databases containing raw location data expose users to a privacy risk (through theft) \parencite{iot, hoh2006enhancing}. Jabbar et al. \parencite{p2p-android} proposes the use of P2P over WIFI and Bluetooth to decrease the need for central instances. Nurminen et al. \parencite{nurminen2006p2p} uses SMS and highlights possibilities to switch to IP once a connection is build up. A decentralized analysis approach and its implications for data privacy is also investigated by Stolpe \parencite{iot} as an alternative to cloud-based IoT.
Kajino et al. \parencite{crowdsourcing} propose an approach for crowdsourcing in which raw data is hidden from the central instance but still aggregated data can be obtained by using encryption methods. Raw data is encrypted using a modified approach of public-private key-pair cryptography in which the sum of two encrypted messages can be decrypted to the sum of the encrypted messages. Furthermore, only a number of messages above a certain threshold can be encrypted this way using the different encryption shares.
%While this approach is close to our work in general, we propose a more general and flexible setup.
Another approach by Hoh et al. \parencite{hoh2006enhancing} also uses encryption in combination with a middleware. The server storing the data and the participant (e.g. a vehicle) share a symmetric key which is stored securely in the vehicle. The middleware ensures the authentication of the participant and forwards the location to the central server without giving away the vehicle's identity. Nevertheless, many researchers \parencite{krumm, twitter, cellphone} have shown that it is easy to infer identity from such data sets and thereby de-anonymize them. 
%Even though they solve the problem of user authentication, it seems hardly feasable as the symmetric key has to be pre-installed on every participating device.

\subsection{Inference Attacks on Published Anonymized Data}
Research has shown, that from a location data set that is pseudonymous, i.e. the identifiers have been stripped off or the data set has been anonymized in another way, it is possible to infer the home location of single users through so-called inference attacks \parencite{krumm, cellphone, privacy-home-work-pairs, hoh2006enhancing, twitter} and also the work location with a slightly lower probability \parencite{cellphone, privacy-home-work-pairs}. The same problem has been identified by Kajino et al. \parencite{crowdsourcing} when using data collected through crowdsourcing.
These home locations or home-work location pairs can then be used to look up the corresponding user's identity e.g. by combining it with publicly available information. One possibility is to reverse code GPS coordinates to addresses and then e.g. search for entries in telephone books to infer the user's identity from its home location \parencite{krumm, privacy-home-work-pairs, hoh2006enhancing}. Especially in suburban areas, this is quite successful as usually one house can be mapped to only one person or family. This identity can then be linked to other sensitive data, e.g. locations visited by the identified user. The same problem also arises in the area of IoT \parencite{iot, hoh2006enhancing}.

\section{Countermeasures to Prevent Inference Attacks}
In general, it has been found by Sweeney et al. \parencite{k-anonymity} that the problem can be solved by providing k-anonymity for the data set. A data set is k-anonymous if querying for an identifier in this data set always returns a result set of at least k entries.
In order to achieve this, several approaches have been investigated. Krumm \parencite{krumm} and Xu et al. \parencite{xu2018location} and Gruteser et al. \parencite{gruteser2003anonymous} propose spatial cloaking. K-anonymity is achieved by dropping data points or perturbing them or dropping all data points around a random point around the home location. Also obfuscating locational data (close to a home location) and mapping GPS points to the next street crossing or only sending e.g. the city or department instead of the exact location is possible to increase anonymity.
More sophisticated approaches as e.g. by Hoh et al. \parencite{time-to-confusion} focus on making it less likely to identify GPS points of one trajectory being subsequent and belonging to the same user.

\section{Limitations of Countermeasures}
Usually, there is a trade-off between the level of anonymity and the usefulness of the data. When k-anonymity is guaranteed, often the resulting data set becomes useless because the data quality is not sufficient anymore \parencite{krumm, cellphone, k-anonymity-old, k-anonymity, k-anonymity-achieving}.
On the one hand, when the data set is tried to be kept useful, data suppression algorithms have only limited success and can only reduce, but not eliminate the risk as shown by Hoh et al. \parencite{hoh2006enhancing}. Research \parencite{time-to-confusion, location-privacy, hoh2006enhancing} finds that anonymization techniques might score well in densely populated areas or areas with high traffic but poorly in sparsely populated areas especially where a single address can be mapped to a single person or family. Also Golle et al. \parencite{privacy-home-work-pairs} find that the applied techniques might not achieve the expected results for individuals whose work and home location are a lot further away than average.
Furthermore, Sweeney \parencite{k-anonymity-achieving} finds that taking other sources and databases into account, k-anonymity might be compromised due to quasi-identifiers e.g. a combination of attributes that do not identify an individual but allow linking two different data sets and by that creating new identifiers. Also, Xu et al. \parencite{xu2018location} find that privacy is lower if road semantics i.e. which types of buildings e.g. hospital, supermarket, are taken into account.
Apparently, the key to a higher degree of privacy also highlighted by Chatzimilioudis et al. \parencite{chatzimilioudis2012crowdsourcing} is to strictly not publish any trajectory.

%\parencite{location-privacy, mix-zones} introduces mix-nodes, that can nevertheless not guarantee privacy and also depnds on a trusted third party.
		%Also \parencite{casper} proposes a solution (close to our summary) how to enable privacy for instant use of location data.

%Decentralized methods for data analysis are also motivated from the area of IoT \parencite{iot}.