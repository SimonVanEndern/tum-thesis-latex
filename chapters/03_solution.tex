% !TeX root = ../main.tex
% Add the above to each chapter to make compiling the PDF easier in some editors.
\chapter{Solution}\label{chapter:solution}


We will use the definition of location privacy as defined by ~\parencite{location-privacy}: " the ability to prevent other parties from learning
one’s current or past location". They further propose a different approach to preserve privacy. TODO!!!


We develop a framework ... 

Beyond the scope of this research is ...

Possibilities for the decentralized analysis are:
\begin{itemize}
	\item Send data via an application e.g. whatsapp or telegram, that takes care of message encryption (trusted third party)
	\item use encryption methods that allow for aggregation as in \parencite{crowdsourcing}
\end{itemize}

A possiblility to infer information about the whole crowd from just a small subset is investigated by \parencite{subset}

Use data formats according to the platform, that shows the standard for each type of data. (Naming conventions, find this website open data, ... something like that)

Blockchain might be useful in this area as well.

What furthermore all papers do not take into account is that when the overall time of sampling increases, the precision of inference attacks will automatically increase as well. 

TODO: Search for papers regarding inference attacks on pure aggregated data

What kind of spatial cloaking to apply? City, region, country? Or purely geometric?

Important: If spatial cloaking request for aggregating data is not successful, it has to be discarded and then retried with a coarser area, but no two overlapping results must be published.

Address the problem of fake users (submitting fake data).

It also seems possible to do the following: 
\begin{itemize}
	\item Store the whole set of venues, etc. of the users home area on the device (in a coarse level). This way the users local data does not have to be revealed at all (and storage space on modern (!!!) phones is sufficient.) It can also be limited / rule based e.g. max. 1 GB data.
	\item install software that directly cloaks the GPS location of the smartphone on the device (and labels it as cloaked). This way services accepting cloaking can respond to this and set a list instead of a single result, services not adhering to this policy will cease to work properly.
	\item Only allow our application to collect (centralized location data).
	\item Limit every applications sending capabilities to a standardized format for authentication (and also for max x times per hour in order to prevent misuse of this template for still exporting data). Most applications do not even require an account to work properly and those who do, can use this template. (e.g. to send emails. For registration purposes an exception has to be implemented). If the application depends on location data, it has to use (and also improve, thus the motivation is there) the publicly available data set.
\end{itemize}
This way we can ensure that either any service only gets a data set of cloaked regions or no data set at all and provide 100\% data privacy. Especially this approach empowers the user, thus the decision to stay private is not anymore based on the goodwill of the service provider but put into the hands of the user itself.

Say how many of the google play store's top 20 applications for location could work without sending any information to the service itself or where inaccurate location data would be sufficient.

For aggregating data as necessary for google road maps, the hugher the passed on data set becomes, the more dangerous? Because it temporarily is  the same as a central server, so malicious users could extract the information from the application.


TODO: cite openStreetMap project as open source, open sourcing data.

Create label for privacy / internationally recognized brand so that applications having this sign are trusted.

Another possibility is to add another set-up: Add a random location data-set to every installation and also do the decentral analysis on this dataset. This way we can also evaluate the results.

It is also possible to use randomly generated location data to test our application: \parencite{brinkhoff2002framework}

TODO: Search for papers regarding mobile crowd-sensing
There are also many references to this in 
\parencite{pouryazdan2017quantifying} 