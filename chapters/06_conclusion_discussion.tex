% !TeX root = ../main.tex
% Add the above to each chapter to make compiling the PDF easier in some editors.
\chapter{Conclusion and Discussion}\label{chapter:conclusion}
\section{Conclusion}

\subsection*{RQ1: How is the aggregation of location data possible without raw data being accessible to anybody but the owner?}
We developed and field tested an approach that allows for decentral aggregation of data that preserves the privacy of users. Due to fast-changing IP-addresses, stable P2P on mobile devices is currently either limited to WIFI and Bluetooth connections or limited by the Bandwidth and cost of SMS. Hence, we proposed a solution based on the assumption of a trusted server that forwards messages between the different Android mobile phones. Nevertheless, the messages are encrypted and only the target device can encrypt the message, as long as the server is not compromised. On the mobile devices, raw data i.e. GPS data, steps and detected activities is collected and locally aggregated through an Android application. This locally aggregated data is then used to serve aggregation requests initiated and forwarded by a central server but the raw data and the locally aggregated data itself is never sent away from the device. Thus, considering the assumption of a trusted server and trusted devices, we propose a solution that indeed manages to aggregate locational data without the raw data being exposed from the collecting devices.
%This highlights that we had to base our solution on the assumption of a trusted server and trusted devices.

\subsection*{RQ2: What types of aggregations can be published?}
We used the proposed setup to compute mean values for the number of steps and the time spent in the activities walking, in a vehicle, biking and running across all aggregation participants in our field testing. In Chapter \ref{chapter:performance} we presented and discussed the results. We found that we had no chance to infer any information about the participating users from the mean values. During field testing, we also collected lists of individual mean values for the average number of steps per day. These lists enable the computation of e.g. median value or distribution function. We have not found any trivial possibility to infer any information about the identity of the user. Furthermore, it does not seem feasible to link values from two subsequent aggregations to the same user. We further showed in Section \ref{inference} that it seems unlikely to infer that the same user participated in two different types of aggregations e.g. the computation of the average number of steps and the computation of the average time spent walking.

\subsection*{RQ3: Are inference attacks similar to inference attacks on anonymized data sets possible?}
We found in Chapter \ref{chapter:performance} that it is possible to infer that the same user participated in different aggregations that overlap regarding the covered timespan. Nevertheless, this is on the one hand based on the fact that the user did not provide data for the days that were only covered by one of both aggregations. On the other hand, we do not see a possibility to infer further information e.g. the identity of the user from such an inference.

\section{Limitations}
As stated in Chapter \ref{chapter:method}, our approach is based on trust among the clients and the server. Nevertheless, while e.g. Kajino et al. \parencite{crowdsourcing} face the same problems in case of a compromised server - namely that the server can create artificial participants and thus obtain the raw values from each user, our setup is more general, allows for more complex aggregations and can be adapted to work via P2P if devices are locally close to each other or P2P generally becomes possible on mobile devices. Also, trust can be established through various other means. Pouryazdan et al. \parencite{pouryazdan2017quantifying} propose a reputation score and Weng et al. \parencite{li2018crowdbc} propose a solution using blockchain technology.

In addition, the field testing should be repeated with more users and more aggregations in order to obtain a larger data set. We do not expect the findings to be any different though, because the key to publication without privacy concerns seems to be that never a GPS location is included in the publication. Only in the question of the aggregation itself, a GPS location might be referenced e.g. by limiting the area of the aggregation as proposed in Chapter \ref{chapter:method}.

Furthermore, it has to be investigated whether additional information available at the server as e.g. the \textit{lastSeen} timestamp of a user could be used to infer and thus leak any further information.

\section{Future Work}
In order to continue this research, the provided setup was carefully designed to allow for very flexible adaptability and extensibility. Some parts proposed in Chapter \ref{chapter:method} were not implemented in this research due to scope limitations. The implementation of these features e.g. the proposed but not yet implemented aggregations, limiting the area of an aggregation and the incorporation of the findings of our field test are a good starting point for future work. In addition, the following Subsections show options for improvements and further research in the respective areas of the Android application and the server itself as well as more conceptual advancements.

\subsection{Android Application}\label{future-android}
There are some promising rather technical improvements to the Android application which are listed as open issues at the public GitHub repository and should be resolved or implemented but were out of the scope of our research. Furthermore, our experience showed that some essentials as e.g. providing a nice information screen to the user when opening the application or showing the own locally aggregated data would boost participation. The aggregations publicly available on the server could also be used to show a comparison of e.g. the personal number of steps to the average or the median. Also making the application available through the play store would be a great relief for the user and simplify the installation process a lot and in addition, enable the roll-out of updates and client-side fixes during field testing. Also the option to delete local data after some time or export it might increase user adoption. Furthermore, the user could be asked to specify his home and work location that is then stored locally. This spares having to locally infer these locations while enabling better aggregations as well as a higher guarantee for privacy protection.

\subsection{Server}
The main possibilities for improvement concerning the server concentrate around dealing with the problem of not available users blocking the aggregation chain. Dynamically selecting the subsequent user at every step and not selecting the whole list of users upfront would allow for benefits from taking the probability of the user being active into account. Nevertheless, no matter when the users are determined, it has to be made sure that the selection of users for an aggregation does not lead to a bias in the aggregation result. Furthermore, encrypting an aggregation not only for one but for several subsequent users\footnote{Once one of these users processes the aggregation, the aggregation requests to the other users become disabled.} would generate some redundancy overhead but yield fewer problems due to not responding users.
Some other rather technical improvements like e.g. pre-populating aggregation requests in order to hinder the first users in the chain from inferring the data of the previous user with a high probability and cleaning the aggregation result from this pre-filling upon completion of the aggregation can be found in a list of issues at the respective GitHub repository as described in Section \ref{future-android}.

\subsection{Related Research Areas}
Our research is very close to some related research areas e.g. decentral computation. Our setup can be used as well as a base for research in this area where problems might be solved locally and collected and added by the server afterwards still with anonymity provided for the users. Similarly, our setup would also allow for pre-populating device databases with artificially generated or elsewhere obtained data e.g. in order to verify whether aggregated results provide the same insights as one could obtain by analyzing a raw data set.
The application can also be modified to be a framework that can easily be incorporated into other apps especially in order to allow for a broader user base in research projects.
Trust is a basic assumption of our framework. A whole area of research e.g. blockchain technology or credit systems or the usage of third parties deal with the question of how to establish or verify trust.

\section{Reproducibility Considerations}
The ReadMe file of the server \parencite{readme-server} provides detailed instructions on how to install and run the server proposed in our work and at which place in the code the URL of the database has to be provided. Section \ref{deployment} gives an example of a deployment option but the software can be installed on any server. The Android application can be installed on any compatible device\footnote{As stated in Section \ref{android-app}, the minimum required API level is 19 and Google Play Services has to be installed on the device.} using the \textit{.apk} file included in the latest version of the Android application \parencite{final-version-app}. Once the server is running, the application automatically registers with the server upon successful installation and granting of location access. Aggregations can be initiated using the API endpoint detailed in Section \ref{api} and Fig. \ref{insert-sample}.
The results obtained should resemble the results obtained in our work. Nevertheless, the actual data will vary accordingly to the activity of the participants.