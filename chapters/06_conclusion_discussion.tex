% !TeX root = ../main.tex
% Add the above to each chapter to make compiling the PDF easier in some editors.
\chapter{Conclusion and Discussion}\label{chapter:conclusion}
\section{Conclusion}

\subsection*{RQ1: Is aggregation of location data possible without raw data being accessible to anybody but the owner?}
We developed and field tested an approach that allows for decentral aggregation of data that preserves the privacy of the users. Raw data i.e. GPS data, steps and detected activity is collected and locally aggregated through an Android application. This locally aggregated data is then used to serve aggregation requests initiated and forwarded by a central server but never send away from the device. Due to P2P not being feasible on mobile devices so far, the server forwards the messages to the different Android mobile phones. Nevertheless, the messages are encrypted and only the target device can encrypt the message. This highlights that we had to base our solution on the assumption of a trusted server and also on trusted devices. Under these assumptions we found a solution that indeed manages to aggregate locational data without the raw data being exposed from the collection devices.

\subsection*{RQ2: What types of aggregations can be published?}
We used the proposed setup to compute mean values for the number of steps and the time spent in the activities walking, in a vehicle, biking and running across all aggregation participants in our field testing. In Chapter \ref{chapter:performance} we presented and discussed the results. We found that we had no chance to infer any information about the participating users from the mean values. During field testing we also collected lists of individual mean values for the average number of steps per day. These lists enable the computation of e.g. median value or distribution function. We have not found a possibility to infer any information about the identity of the user. Furthermore, there is no possibility to link values from two subsequent aggregations to the same user. We further showed in Section \ref{inference} that it seems unlikely to infer that the same user participated in two different types of aggregations e.g. the computation of the average number of steps and the computation of the average time spent walking.

\subsection*{RQ3: What is the risk of inference attacks on aggregated data due to overlappings in the covered timespan?}
We found in Chapter \ref{chapter:performance} that it is possible to infer that the same user participated in different aggregations that overlap regarding the covered timespan. Nevertheless, this is on the one hand based on the fact that the user did not provide data for the days that were only covered by one of both aggregations. On the other hand, we do not see a possibility to infer further information e.g. the identity of the user from this inference.

\section{Limitations}
As stated in Chapter \ref{chapter:method}, our approach is based on trust among the clients and the server. Nevertheless, while e.g. \parencite{crowdsourcing} face the same problems in case of a compromised server - namely that the server can create artificial participants and thus obtain the raw values from each user, our setup is more general, allows for more complex aggregations and can be adapted to work via P2P if devices are locally close to each other or P2P generally becomes possible on mobile devices. Trust can also be established through something like XXX.

In addition, the field testing should be repeated with more users and more aggregations in order to obtain a larger data-set. We do not expect the findings to be any different though because the key to publication without privacy concern seems to be that never a GPS location is included in the publication. Only in the question of the aggregation itself a GPS location might be referenced e.g. by limiting the area as proposed in Chapter \ref{chapter:method}.

\section{Future Work}
In order to follow with this research, we provide the setup to easily implement and test further aggregations in future research in order to further support our hypothesis.
Research based on our work has many options to improve our work.

\subsection{Android application}
Regarding the Android application, there is a list of rather technical open issues at the GitHub repository, that should be resolved but were out of the scope of our research. Furthermore, our experience showed that some essentials as e.g. providing a nice information screen to the user or showing him his own locally aggregated data would ease the participation. One could also use the aggregations to show them a comparison of their number of steps, etc. to the average. Also making the application available through the play store would be a great relief for the user and simplify the installation process a lot and in addition enable the roll-out of updates and client-side fixes during deployment / field testing. Also the option to delete local data after some time or export it might increase user adoption. Furthermore, the user could be asked to input his / her home and work location which is stored locally in order to avoid having to infer it and give the app more possibility to protect privacy.
Change of public / private key (as in mix-nodes) after some time.
(Put somewhere else!! TODO) The app can send traffic alerts to the server if it is on a route where usually traffic is far faster. (This also via other nodes in order to not letting the server know who is on this route.)

\subsection{Server}
Improve the rollback when a user does not respond, make request available to different users and make user selection dynamic. Also prepopulation of requests. When dynamically adding users to the aggregation request, we need another field for setting the limit so that when the limit is reached, the user sends back the result.

\subsection{Chapter 3 complete implementation}
Implement / change the aggregation from mean to list for all types. Implement the other proposed aggregations. Enable the proposed solution for restricting aggregation areas. Dynamically generate nextUser list.
Evaluate how many people have to participate in an aggregation request to be representative / how to chose users participating in it in order to not be biased (e.g. when taking always the most recent active users, the users only online a few times are day are discriminated against)

\subsection{Improvement of the framework itself}
Our framework would also allow to pre-populate (simulated) smartphones with artificially generated or otherhow collected data in order to test and verify the functionality.
Another use of our framework is the area of decentralized computation. Problems might be solved locally and collected by the server afterwards and the pieces put together still with anonymity for the users.
The application can be modified to be a framework that can easily be incorporated into other apps. Also, the approach could easily integrated into existing projects like open maps or be build modular in order to allow using it as a library with other applications.

\subsection{Trust issue}
Verify public key e.g. through telephone number passing and verifying this way and then building a network of verifications. It could also be implemented to verify another public key but requesting an SMS, .... (harder with changing public-private key-pairs)

\section{Reproducability Considerations}
The ReadMe file of the server \parencite{readme-server} provides detailed instructions of how to install and run the server proposed in our work and where in the code the url of the database has to be provided. Section \ref{deployment} gives an example of an deployment option but the software can be installed on any server. The Android application can be installed on any compatible device\footnote{As stated in Section \ref{android-app}, the minimum required API level is 19 and Google Play Services has to be installed on the device.} using the \textit{.apk} file included in the realease \parencite{final-version-app}. Once the server is running, the application automatically registers with the server upon successful installation and granting of location access. Aggregations can be initiated using the API endpoint detailed in Section \ref{api} and Fig. \ref{insert-sample}.
The results obtained should resemble the results obtained in our work. Nevertheless, the actual data will vary accordingly to the activity of the participants.