% !TeX root = ../main.tex
% Add the above to each chapter to make compiling the PDF easier in some editors.
\chapter{Performance and Evaluation}\label{chapter:performance}
\section{Deployment}
The proposed Android application and server have been tested on 16 devices for one week from 30.05.2019 until 06.06.2019. The server was deployed\footnote{The version deployed during field testing can be found here: https://github.com/SimonVanEndern/location-server/releases/tag/v1.0 The final version includes the same functionality but includes improvements (e.g. bug-fixes, comments, ...) over the deployed version.} in the IBM Cloud as an 128 MB node.js instance. The database was hosted as a free version at mongodb.com.
We ran each of 8 requests on a daily basis and also for each timespan of several days within this period. The raw results can be found at XXX\footnote{Only 7 of the 8 aggregations are available as it is. The results of aggregation 8 where modified in order to protect user priacy}. We used this testing period also to improve the performance of the server as well as the Android application and to find and remove bugs.

\section{Performance}
The application used very few data (qualitative analysis attached in the appendix) and also the battery consumption was not two high (also qualitative).

\section{Results}
We computed the aggregated results for each of the 6 days of the testing period separately and for 5 timespans from each of those days until the last day. The results can be seen in figure XXX. 
On most of the days the average time spent walking is roughly around one hour per day. Nevertheless, on the 3rd of June the value is clearly higher and 31st as well. We do not see any reason for this spike -- 03.06. is a regular work day - nevertheless, the value is not that high as it would suggest errors. The value on the first day being below the other values is definitely attributable to the fact that we started to roll out the application on this day.
The average time running is around 1-2 minutes per day. This is not surprising. The only scenarios usually are when somebody has to catch some transport or actively is running (and might probably not carry his or her phone).
Due to an error in the setup, we only have the daily average data for the time spent biking which ranges from 11 to 31 minutes per day and sounds realistic regarding that the participants where all in Munich.
The time spent in a  vicle ranges from half an hour to 90 minutes on average per day and has one spike of almost 2 hours on 31.06. This can be explained as one of our research team had a very long car ride on this day. Comparing the data with the registered trajectories, there is another pretty long car ride which might have attributed to the high average of 90 minutes per day. 
The average steps per day range in the lower range of some thousands. Though, there are some values which clearly expose an error in our aggregation process. Excluding these definitely incorrect values from the statistics, the values all are in an expected range. 
Investigating the list of average steps, it is clear that the erro did not occur in the aggregation process but in the process of local aggregation of data. There is always maximum one value off, all other values are withinn an expected range. The zero values are due to non step sensor being present on the respective phones (which are excluded in the average aggregation).
Using the algorithm described in XX a total of 406 trajectories where computed out of the raw GPS data. A part of the trajectories are shown in figure XX. The complete results can be found in the appendix. \footnote{Whenever the trajectory started or ended clearly in a precise private location e.g. housing, we modified this location. So the results actually do not represent real trajectories anymore} 

\section{Implications}
Our results show clearly, that there is no privacy risk imposed on the user upon publication of the aggregated average data. Especially it was shown that when repeating the request, the user base changes and the value accordingly. So, publishing pure mean values of analysis does not impose any privacy risk to the user. Furthermore we showed exemplary with the steps aggregation, that also the collection of the users' average values is possible without posing a privacy risk to the user. This implies, that not only the aggregation of mean values but also the collection of locally aggregated data is possible without privacy concerns. We do not see any possibility to infer that the same user took part in different aggregations. There might be a chance in case of a very high steps value and a very high mean time spent walking or running, but this would only allow to assume the same user being present in both aggregations which allows for no further inference.

Nevetheless when collecting the average data, through overlapping requests e.g. one from 30.05 - 02.06 and one from 01.06. - 04.06 it might be possible to infer the exact date of some of the data present in both requests. Though, we do not see any risk aggregating on a daily basis or even hourly if there is any reason for this granularity level. Thus, the inference does no harm.

The experimental collection of trajectories clearly shows privacy risks as pointed out in XX. Nevertheless, the data suggests that those trajectories can easily be linked locally and identify changing of transport system or e.g. metro line. For example at the locations Giselastraße, Odeonsplatz and central station and other locations, many trajectories end respectively start. Mapping the current activity to those trajectories enables aggregations as mentioned in XX. Noting that locally all data points of the trajectories are available, it is also easy to compute the distance travelled. Also it is possible to compute how many people combine e.g. bike and car or public transport and furthermore identify, which station is most likely (in case of public transport) to be combined with bike.

Also it would be possible to create a "travelled road" or "travelled public transport" map by aggregating the trajectories data so that there would not be multiple trajectories but each road either marked used or not used. This way it would be possible to identify that a user lives in a certain area but could not link to the work location due to obfuscation with other routes joining. This map would on the other hand allow to have a feeling of which areas are covered by the data / app. This map could also be extended with the average speed on the respective road depending on vehicle or bike and also compare whether bike is faster.

Computing time to work and average time at work.

It also allows for computing how many people go by car, bike, or mixed to work.

We also started an aggregation a second time - walking from 30.05 - 01.06. which shows as expected a different result than the original aggregation due to the 10 users being not the same users as in the first request. Nevertheless, the value is not far off as expected.

\section{Scenario Traffic Jam}
A typical scenario of google maps is o notify users about traffic jams and suggest alternate routes. The calculation of alternate routes taking traffic jams into account can clearly happen locally with the maps data. Google maps works when offline. The data about all current traffic jams can also be made publicly available through a server. Generating the data can also happen without exposing raw data: The user downloads a map containing data of the usual speeds at each street. While the user is driving, the app registers the speed and compares it in the background to the normal speed. If the speed is significantly lower, the user chooses a random list of known users and sends the signal as a request for those users to the server. They randomly according to a fixed percentage choose to inform the server about the traffic jam or forward the signal another time. The signal contains a unique id thus that the server even when receiving it muliply times knows it is from one user. If more than a threshold of signals is received, a traffic jam is "created". Also the request is not forwarded anymore after a certain time to stop it from spreading unlimited.

\section{Evaluation}
The results support our hypothesis that data can be analyzed decentrally and that aggregated data can be published without any privacy concerns. The aggregations of mean values clearly leave no doubt about full privacy protection as there even is no personal data involved anymore. The listing of mean values of average number of steps per participant allows for more advanced statistical analysis while at the same time the values cannot be mapped to persons. Even when conducting the same request twice, due to users being chosen dynamically, one could most probably see if the same user participated in the second request but nothing else. When aggregating over another time period (that might have an intersection with the other one), there is not even the chance to identify whether the same user participated in both aggregations. As requests are started not at the same time (TODO!!!), and users are in a future setting allocated dynamically to the request, there is also no chance to link the data from different aggregations. From the number of steps one could infer the time somebody spent walking, but as it is not given whether the steps where conducting walking, running or both, this linking would result in a very poor performance and also only reveal that to a very low probability, the user with X steps in one aggregatioin is the same user spending X minutes walking the same day. 
The listing of trajectories created a dataset that clearly shows the vulnerability highlighted in XX and XX. Nevertheless, this is no aggregation but just a collection of raw data with stripped of identifiers and timestamps anonymized to a daily basis. The results show clearly that our setup is sufficiently accurate to field test the other aggregations proposed in XX and further prove our thesis. The data shows, that e.g. A change of transportation system can clearly be identified. 