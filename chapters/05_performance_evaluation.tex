% !TeX root = ../main.tex
% Add the above to each chapter to make compiling the PDF easier in some editors.
\chapter{Performance and Evaluation}\label{chapter:performance}
\section{Deployment}
The proposed Android application and server have been tested on 16 devices for one week from 30.05.2019 until 06.06.2019. The server was deployed\footnote{The version deployed during field testing can be found here: https://github.com/SimonVanEndern/location-server/releases/tag/v1.0 The final version includes the same functionality but includes improvements (e.g. bug-fixes, comments, ...) over the deployed version.} in the IBM Cloud as an 128 MB node.js instance. The database was hosted as a free version at mongodb.com.
We ran each of 8 requests on a daily basis and also for each timespan of several days within this period. The raw results can be found at XXX\footnote{Only 7 of the 8 aggregations are available as it is. The results of aggregation 8 where modified in order to protect user priacy}. We used this testing period also to improve the performance of the server as well as the Android application and to find and remove bugs.

\section{Results}
The results support our hypothesis that data can be analyzed decentrally and that aggregated data can be published without any privacy concerns. The aggregations of mean values clearly leave no doubt about full privacy protection as there even is no personal data involved anymore. The listing of mean values of average number of steps per participant allows for more advanced statistical analysis while at the same time the values cannot be mapped to persons. Even when conducting the same request twice, due to users being chosen dynamically, one could most probably see if the same user participated in the second request but nothing else. When aggregating over another time period (that might have an intersection with the other one), there is not even the chance to identify whether the same user participated in both aggregations. As requests are started not at the same time (TODO!!!), and users are in a future setting allocated dynamically to the request, there is also no chance to link the data from different aggregations. From the number of steps one could infer the time somebody spent walking, but as it is not given whether the steps where conducting walking, running or both, this linking would result in a very poor performance and also only reveal that to a very low probability, the user with X steps in one aggregatioin is the same user spending X minutes walking the same day. 
The listing of trajectories created a dataset that clearly shows the vulnerability highlighted in XX and XX. Nevertheless, this is no aggregation but just a collection of raw data with stripped of identifiers and timestamps anonymized to a daily basis. The results show clearly that our setup is sufficiently accurate to field test the other aggregations proposed in XX and further prove our thesis. The data shows, that e.g. A change of transportation system can clearly be identified. 