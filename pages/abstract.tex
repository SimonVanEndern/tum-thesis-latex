\chapter{\abstractname}

A large fraction of GPS data-sets are not available to the public but proprietory to the companies collecting them e.g. Google. The public might benefit from access to these data-sets. E.g. cities might better understand traffic problems and come up with innovative solutions. Nevertheless, the publication of these data-sets is not wanted in most cases and also not possible without severly intruding the privacy of the people the data was collected from. While research has investigated possibilities to anonymize the data-sets to enable publication the overall problem has not been solved yet, especially does it depend on the goodwill of companies owning these data-sets. To solve this, a crowdsourcing approach for Android smartphones has been developed and tested. This approach decentrally aggregates location data and publishes only aggregated data in order to preserve users' privacy. Due to P2P not generally being possible on mobile devices, the solution depends on the assumption of a trusted server and devices, respectively that the developed source code runs as it is on the server and the devices. In a field testing it was shown that the proposed solution works. Furthermore, the results of the field test with 16 participants showed that aggregations as the mean value, median value and distribution function seem to be possible to be published without violating the users' privacy. Furthermore, it was shown that more complex aggregations as e.g. identifying which subway stations are most suitable for combining biking and public transport seem possible with the developed setup and the resulting data quality.